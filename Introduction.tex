\section{Introduction}
Human Monkeypox (MPX) is an emerging zoonotic disease endemic to forested regions of central and western Africa. The MPX virus is a member of the Poxviridae family, in the genus Orthopoxvirus \cite{Kulesh2004}. Since the eradication of smallpox, MPX is considered the most important orthopoxvirus with respect to human infection \cite{Khodakevich1988, WorldHealthOraganization1980, Fine1988}.

MPX was recognized as a distinct human disease in 1970 during smallpox eradication efforts in the Democratic Republic of Congo (DRC) when its continued presence was confirmed in rural, forested areas.5,6 After the global cessation of smallpox vaccination in 1980, there were concerns that MPX could emerge from central Africa to replace smallpox in the absence of vaccination programs, and that increased surveillance was necessary to assess this risk.3,7 As a result, the World Health Organization (WHO) initiated the first active MPX surveillance program in the DRC from 1981-1986 to better understand the public health significance of MPX as the population with cross-protective immunity from smallpox vaccination decreased.3 Epidemiological data and stochastic models of human-to-human transmission produced from this study concluded that the reproductive number ($R_{0}$) of human MPX was less than 1, with most cases believed to have been infected through contact with forest animals. Viral evolution would be necessary for sustained human-to-human transmission and endemicity.4,8-11 However, more recent studies have raised concerns about both the long-term validity of these conclusions.7 

During active surveillance carried out between 2005 and 2007, researchers found that annual incidence of MPX within a forested health zone, Kole, increased 20-fold compared to the 1980’s WHO surveillance in the same zone (0.72 to 14.42 per 10,000).7 The increase is potentially influenced by many factors, such as the decreasing population carrying cross-protective immunity from smallpox vaccination, mass human migration into endemic forested regions due to prolonged conflict, a subsequent increased reliance on bushmeat as a main source of protein and thus increased exposure to potential animal reservoirs.7,14,15 Disease surveillance systems have improved since the war ended in 2002 and have contributed to the increased reporting of suspected cases, however, a recent study has shown that there was a 4-fold increase in MPX cases between 2001-2013, independent of improved surveillance and reporting systems.14 

Studies since 1981 report that the vaccinia (smallpox) vaccine confers 80-85\% protection for MPX, leaving vaccinated individuals 5.2-fold less likely to acquire the disease than those unvaccinated.4,7 Though vaccine-derived immunity is thought to be long-lasting, only 24.5\% of the current population located in surveillance regions have received the vaccine, compared to 84.7\% in 1981, thus drastically reducing herd immunity and increasing the proportion of susceptible individuals.7,19,25 

Epidemiological curves depicting outbreaks of human MPX are typically bi-modal. Frequency of primary cases peaks after introduction of a point source of infection and is separated from the peak of secondary cases based on the length of “rash-to-rash interval,” or the average incubation period, which is believed to be 12 days (range, 7-21) (Figure 2).9,26 Individuals are believed to be infectious from the time of rash onset, until desquamation, usually 2-4 weeks. Transmission can occur through direct contact, and less efficiently, through respiratory droplets.27 

In the WHO sponsored active surveillance program from 1981-1986, 338 cases were investigated, with 245 (72\%) considered primary or co-primary, and 93 (28\%) secondary, assumed human-derived cases. The secondary attack rate among 431 unvaccinated household contacts was 9.3\%.19 The attack rate among contacts of secondary cases was only 1.3\%, meaning that most chains of transmission died out after 1-2 generations, with the exception of one chain which lasted for four generations.8 

The crude absolute  secondary attack rate was found to be 0.030, with the basic reproduction rate (R0) averaging 0.815.4 However, it should be noted that this R0 was calculated using the unvaccinated attack rates, 0.110 and 0.038 respectively.4 Therefore, it may be assumed that this would be the R0 in a completely susceptible population.29 To estimate the effective reproductive number (Reff), which accounts for the difference in attack rate between vaccinated and unvaccinated contacts and is more representative to the actual transmission potential at the time of the study, one should use the overall attack rates for household and non-household contacts, 0.043 and 0.015 respectively (Table 1). This method determined Reff to be 0.319, which represents the average number of secondary cases caused by each case, accounting for differences in vaccination status. 

More recently, a chain of transmission containing seven generations of secondary cases was observed in the Republic of Congo in 2003,33 suggesting that human-to-human transmission is increasing. It is difficult to ascertain whether this increase is due to a true increase in virus transmissibility (R0) as a result of viral adaptation, or other factors that could increase incidence without changing R0, such as increased frequency of introduction from animal reservoirs, or simply the result of an increased proportion of the population being susceptible to infection as a result of smallpox vaccination cessation.29 In the case of the latter option, longer chains of transmission would be observed among susceptible household contacts, but would die out after this close population is depleted. This possibility would obscure the detection of a true increase in R0.29

[Jezek Model] Although the authors concluded that there was a limited potential for MPX to be sustained in human populations, they did not take into account the large-scale anthropogenic changes which occurred during the last two decades, such as prolonged conflict, mass human migration into forested areas, and increased reliance on bushmeat. This means that their critical assumption that the annual rate of introduction from animal reservoirs would remain low at 0.35 per 100,000 is most likely flawed. Additionally, the model does not take into account the possibility of viral evolution and adaptation, which would increase its ability to transmit between human hosts.34

A more recent study argues that stochastic techniques cannot adequately model infectious diseases characterized by “stuttering chains,” which have a non-zero R0 (0< R0<1), including diseases close to elimination, and many types of zoonotic infections. Rather, models for these diseases should include measures of dispersion, or heterogeneity in transmission dynamics, which is crucial for reliably detecting a statistically significant change in transmissibility.36 A model for secondary transmission using a two-parameter negative binomial distribution with a mean R0 and a dispersion parameter k determined that it is possible to robustly simulate outbreaks using infection chain size and accurate primary case data. Results found R0 to be 0.30 (95\% CI: 0.21-0.42) and k to be 0.33 (95\% CI: 0.17-0.75), which supports previous conclusions that MPX will have limited endemic ability without viral adaptation, even given the highly heterogeneous transmission dynamics of MPX.36

This study builds on previous analyses [of the active surveillance study] by incorporating contact tracing information and geographical information to conduct a spatiotemporal analysis of disease transmission. We classified cases as primary, co-primary, or secondary, and we compared the characteristics and frequencies of human transmission from primary to secondary cases. With this information, we determined the difference in R0 between the 1980s study and the current one. 

