\section{Results}

\subsection*{Monkeypox Transmission Dynamics}
Of the 293 confirmed MPX cases reporting after contacts, 87 individuals (29.2\%) started a chain of transmission, infecting 96 individual secondary cases, who represented 166 reported contacts due to their contact with multiple cases. Each contact was listed by an average of 1.28 cases (Table 4). The crude absolute secondary attack rate was found to be 0.067 (Table 5).

\begin{table} 
    \begin{tabular}{ c c c c c c c c c }
         & 1981-1986 &  &  &  & 2005-2007 &  &  &  \\ 
         & No . Cases & No contacts & AR Crude & Reff & No cases & No contacts & RAR Crude & Reff \\ 
        total contacts reported &  &  &  &  &  &  &  &  \\ 
        individual contacts &  &  &  &  &  &  &  &  \\ 
    \end{tabular} 
    \caption{Table 1. Differences between crude secondary attack rates (ARcrude) and effective reproduction number (Reff) between the active surveillance studies taking place 1981-1986, and 2005-2007. The data from the 1980s is reported as total contact reports, while the more recent study was able to identify individuals who were reported as a contact of multiple cases} 
\end{table}


\begin{table}[!hb]
        \begin{tabular}{lcccccccc} 
        \toprule
         & \multicolumn{4}{c}{1981-1986} & \multicolumn{4}{c}{2005-2007} \\ % Amalgamating several columns into one cell is done using the \multicolumn command as seen on this line
\cmidrule(r){2-5} % Horizontal line spanning less than the full width of the table - you can add (r) or (l) just before the opening curly bracket to shorten the rule on the left or right side
\cmidrule(l){6-9}
 & No. Cases & No. Contacts & AR crude & Reff & No. Cases & No. Contacts & AR crude & Reff\\ 
\midrule % In-table horizontal line
Total contacts reported & 3686 & 93 & 0.025 & 0.275 & 1861 & 166 & 0.089 & 0.563\\ 
Individuals contacts & - & - & - & - & 1453 & 97 & 0.067 & 0.485\\ 
\bottomrule
        \end{tabular}
           \caption{Table 1. Differences between crude secondary attack rates (ARcrude) and effective reproduction number (Reff) between the active surveillance studies taking place 1981-1986, and 2005-2007. The data from the 1980s is reported as total contact reports, while the more recent study was able to identify individuals who were reported as a contact of multiple cases.}
     \label{tab:table}
      \end{table}