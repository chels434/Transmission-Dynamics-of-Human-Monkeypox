Chains were stratified by whether they contained co-primary cases or not to determine whether there was a significant difference in chain size distribution between chains with co-primaries and chains with just a single primary case. Overall, primary and co-primary cases made up 72.6\% of total cases, and 77.4\% of chains did not spread to secondary generations. Approximately 19.6\% of chains contained one or more co-primary cases, of which 65.9\% did not spread to subsequent generations. This is significantly different than the chains that contained only a single primary case, 80.2\% of which did not spread to secondary generations (Figure 5). Overall, the average chain size was 1.76 cases (range, 1-13) and the longest chain reached six generations. Chains with one primary case contained an average of 1.42 cases (range, 1-13), as opposed to those with co-primaries with 3.14 cases (range, 2-12). 