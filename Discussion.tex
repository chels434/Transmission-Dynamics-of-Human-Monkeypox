Since the first active MPX surveillance study in the 1980s, annual incidence in parts of the Sankuru district has increased more than 20-fold.7 Although researchers in the 1980s determined that there was a low potential for epidemic threat by MPX, this conclusion is now questionable due to anthropometric changes in human contact with animals and other humans, declining vaccination coverage, and possible viral adaptation to humans. This study aimed to determine whether the 1980s Reff and the current Reff are significantly different.

The proportion of primary and co-primary cases versus secondary cases was nearly identical to that of the 1980s, at 72.6\% and 27.4\% respectively. However, we found evidence to support that inadequate contact tracing and follow up of cases could have led to an overestimation of primary cases. Many chains are likely to contain additional suspected cases who were investigated within the appropriate incubation period, but did not have samples collected or tested. This would occur as a result of late follow-up visits or inadequate supplies needed to collect samples. As Blumberg et al (2013) have reported, while changes in Reff can be estimated from chain size and k alone, it relies on the assumption that the correct proportion of primary cases per chain is known.36 Contact tracing helps to connect primary cases to subsequent generations of cases that may have been missed, thus increasing the robustness of model. 

Chain size ranged from 1 to 13 cases, with an average size of 5.3. The longest chain reached six generations and consisted of 13 cases, including one vaccinated woman over the age of 60.

Using data from the 1980s surveillance period, Fine et al (1988) reported that the R0 of MPX was 0.815. However, this was found using only the attack rates of unvaccinated household and non-household contacts. When the overall secondary attack rates were substituted into the equation, Reff was found to be 0.319. We determined that the Reff during the 2005-2007 surveillance period was either 0.419 when calculated with the individual contact attack rates, and 0.570 when calculated with the attack rate of total contacts reported We calculated both of these values in order to determine the high and low estimates of Reff, as the individual attack rate assumes that each individual was infected by just one case while the total contact attack rate treats contacts reported by multiple cases as if each report represented a unique individual. The true Reff is likely to be between these two values; however, it is impossible to determine which case infected contacts listed by multiple cases. The 1980s Reff was calculated using total reported contacts rather than individuals; thus, for the purpose of comparison, we used our upper limit, 0.570. 

Although the Reff is still less than 1, this study shows that the human-to-human transmissibility of MPX virus has increased since the WHO surveillance period in 1981-1986. It is likely that an increased rate of zoonotic introduction, along with decreasing vaccination coverage is contributing to the increased incidence seen over the past 30 years. This is of great interest to public health, as it is recognized that poxviruses are readily able to infect and adapt to naïve species. It is suspected that smallpox originated in an animal species, and slowly transitioned to humans as its only host species. Along the way, the virus is thought to have shed parts of its genome unnecessary for human transmission, as it is the shortest of the orthopoxviruses.18 This is applicable to the epidemiology of MPX today since it has been found that certain clades sequenced from the 2005-07 study have both lost parts of their genome and are more infectious or severe. It is possible that MPX could follow a similar route to its ancestor smallpox; however, it would most likely take centuries to become an exclusively human virus. For this reason, the current MPX virus would be more difficult to eradicate than smallpox due to its wide range of potential host species. The best course of action would therefore be to install control mechanisms to prevent or delay further adaption to humans.

