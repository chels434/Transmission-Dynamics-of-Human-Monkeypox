\section{Methods}

From 2005 to 2007, the UCLA-DRC Research Group, in coordination with the DRC’s Ministry of Health (MOH), facilitated an active surveillance program to identify cases of human MPX in the Sankuru district of Kasai Oriental province, in central DRC, the same region as the 1980’s WHO program. The active surveillance protocol and results are reported in Rimoin et al \cite{Rimoin2010}.

Active surveillance took place within 9 health zones of the Sankuru district of Kasai Oriental province in the DRC (Figure 3). The Sankuru has an area of 105,378 km2 and is comprised of dense canopy rainforest, mosaic/gallery forest and savannah \cite{Fuller2011}. The most populous cities, Lodja, Kole and Lomela are connected to supply roads and waterways; however, most villages contain fewer than 1000 people, are difficult to access, and are established in small forest clearings or traditional agricultural land on the forest edge \cite{Rimoin2010}. In 2007, the 9 health zones had an estimated population of 676,839.37 The inhabitants typically rely on subsistence farming and hunting to provide nutrients, with virtually all protein obtained from hunted wildlife, the most common sources being duikers, monkeys, and rodents.2,25,38,39 Since 2001, this area of the country has consistently reported the largest incidence of MPX cases and is thus considered an endemic region for MPX \cite{Rimoin2010}. 






