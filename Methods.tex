\section{Methods}

From 2005 to 2007, the UCLA-DRC Research Group, in coordination with the DRC’s Ministry of Health (MOH), facilitated an active surveillance program to identify cases of human MPX in the Sankuru district of Kasai Oriental province - the same region as the 1980s WHO program. Since 2001, this area of the country has consistently reported the largest incidence of MPX cases and is thus considered an endemic region for MPX \cite{Rimoin2010}. The active surveillance protocol and results are reported in Rimoin et al \cite{Rimoin2010}.

Since 2010, MPX has been on the Ministry of Health's list of nationally reportable diseases in the DRC. Thus, all suspected MPX cases in DRC are primarily investigated as mandated by the national program for passive MPX disease surveillance. Consenting cases investigated in the targeted 9 health zones of the Sankuru district who met the  Ministy of Health’s case definition (fever ≥38 ºC, and a vesiculopustular rash), were recruited to participate in this study from 23 November 2005 – 31 November 2007. 

The MPX surveillance officer in each health zone conducted a clinical examination on all suspected cases, including the collection of crusted scabs and vesicle fluid from active cases, and administered an oral questionnaire in the local dialect, to collect socio-demographic, clinical and exposure data. GPS coordinates were collected for approximately 58\% of cases investigated. 


At the time of case investigation or during a follow-up visit, patients were asked to complete a contact tracing questionnaire to determine whether they had contact with other individuals displaying a similar illness in the three weeks prior to disease onset  (\textit{before contacts}). Patients were also asked to report all of their all sick or healthy contacts in the weeks after disease onset (\textit{after contacts}).

All contacts with unique names were assigned a new, second identification number after data entry to protect their identity, and to link individuals who had been assigned multiple ID codes due to typo or data collection error. Fuzzy matches and similar names were identified using n-gram clustering algorithms provided by the OpenRefine software program. Contacts in each cluster were determined to be the same or different individuals by comparing the associated data collected in each reporting instance. Variables taken into account were the health zone, village, name of the head of household, relationship to the case, and dates of contact. If non-unique names had differing associated data, they were determined to be separate individuals, and were re-assigned new unique IDs. If contacts with unique but similar names were found to have matching associated data, they were re-assigned to the same unique ID.

The health care worker investigating the case was asked to record the case ID of reported contacts who had previously been investigated as cases, and to update this information if they presented symptoms at a later date. However, as not all cases received a follow-up visit to update this information, we used the same process as described above to compare the names of all contacts to the names of all investigated cases, to determine all the reported contacts who were also investigated as cases.


Cases investigated during the study were included in this analysis if their biological sample tested positive for MPX, as determined by PCR assays. 

Ethical approval for this study was obtained from participating institutions (University of California and the Kinshasa School of Public Health), and informed consent was obtained from all participants. All researchers completed required trainings and were approved as key personnel by the institutions. 




