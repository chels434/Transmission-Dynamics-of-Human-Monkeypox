\section{Methods}

From 2005 to 2007, the UCLA-DRC Research Group, in coordination with the DRC’s Ministry of Health (MOH), facilitated an active surveillance program to identify cases of human MPX in the Sankuru district of Kasai Oriental province - the same region as the 1980s WHO program. The active surveillance protocol and results are reported in Rimoin et al \cite{Rimoin2010}. Since 2001, this area of the country has consistently reported the largest incidence of MPX cases and is thus considered an endemic region for MPX \cite{Rimoin2010}. 

Data from the 1980s WHO active surveillance program were collected from 1981-1986, and were reported in Jezek et al \cite{Jezek1988}.

Since 2010, MPX has been on the Ministry of Health's list of nationally reportable diseases in the DRC. Thus, all suspected MPX cases in DRC are primarily investigated as mandated by the national program for passive MPX disease surveillance. Consenting cases investigated in the targeted 9 health zones of the Sankuru district who met the  Ministy of Health’s case definition (fever≥38 ºC, and a vesiculopustular rash), were recruited to participate in this study from 23 November 2005 – 31 November 2007. 

Ethical approval for this study was obtained from participating institutions (University of California and the Kinshasa School of Public Health), and informed consent was obtained from all participants. All researchers completed required trainings and were approved as key personnel by the institutions. 

\subsection{Definitions}
Similar to algorithms described previously,4,9,41 an investigated \textit{source case} was classified as \textit{primary} if it was the first case investigated in an independent outbreak chain, or had no history of contact with a symptomatic human within three weeks before rash onset. Primary cases were assumed to be caused by an animal source. \textit{Onset} is defined as the first day of rash presentation and is considered the first day of infectiousness. Investigated source cases were asked to report all \textit{contacts}, defined as individuals coming into direct contact with a case, in the three weeks after onset. As individuals may have been reported as contacts of multiple cases, each report was considered a \textit{contact event}. \textit{Household contacts} were defined as those residing in the same household as the source case.

A \textit{chain of transmission} includes the primary case, and all subsequent cases stemming from this initial infection. A \textit{cluster} of cases may contain chains arising from multiple primary cases.29 Cases in the same chain or cluster with an onset less than seven days after exposure to the primary case were classified as \textit{co-primary case}s. Any cases with an onset 7-34 days after reported contact with a symptomatic primary case was classified as a \textit{secondary case} of the first generation, and assumed to be caused by a human source. The \textit{generation} of secondary cases was determined by their location within the temporally classified chains of transmission.29

\textit{Secondary attack rates} are defined as the proportion of new cases among contact events, within the known incubation period \cite{Dixon2015}. Attack rates can be compared to the \textit{effective reproductive number} ($R_{eff}$), which is defined as the average number of secondary cases caused by each individual case, with regard to population susceptibility.10,27 This is as opposed to the \textit{basic reproductive number} ($R_{0}$), which is the average number of secondary cases caused by each case in a completely susceptible population.29 

\subsection{Data Collection and Management}
The MPX surveillance officer in each health zone conducted a clinical examination on all suspected cases,
including the collection of crusted scabs and vesicle fluid from active cases and administration of an oral questionnaire in the local dialect, to collect socio-demographic, clinical and exposure data, contact tracing data and GPS coordinates.

At the time of case investigation or during a follow-up visit, patients were interviewed to complete a contact tracing questionnaire reporting contact with symptomatic individuals in the three weeks prior to disease onset  (\textit{before contacts}) and all contacts in the weeks after disease onset (\textit{after contacts}). If a reported contact had previously been investigated as a case, or was investigated at a later date after developing symptoms, the investigating health care worker was asked to record the contact's case ID, and to update this information during follow-up visits to each village. Contact tracing information was recorded on paper and entered into a custom relational database using FileMaker Pro v13 at a later date.

All unique names of both cases and contacts were assigned a new identification code after data entry to protect their identity, and to link the records of identified individuals who had been assigned multiple unique IDs due to typos or data collection errors. Fuzzy matches and similar names were identified using n-gram clustering algorithms provided by the OpenRefine software program. Contacts in each cluster were determined to be the same or different individuals by comparing the associated data collected in each reporting instance. Variables taken into account included the health zone, village, name of the head of household, relationship to the case, and dates of contact. If non-unique names had differing associated data, they were determined to be separate individuals, and were re-assigned new unique IDs. If contacts with unique but similar names were found to have matching associated data, they were re-assigned to the same unique ID **[diagram to show flow?]. The names of all contacts were also compared to the names of all investigated cases, to link any contacts who were also investigated as cases to their case data, regardless of whether their case ID was provided in the contact tracing questionnaire. 

The datasets containing case information and contact tracing information were merged to create a dataset of contact events. Individuals listed as before contacts that had been investigated as cases and all investigated cases reporting after contacts were classified as \textit{sources}, while all individuals listed as \textit{after contacts} and investigated cases reporting \textit{before contacts} were classified as \textit{sinks}.

Within the current study, we used the data on relationship to the case and activities performed with the case to determine whether certain contacts or certain activities had significantly higher risk of becoming infected.

\subsection{Descriptive Analysis}
We performed a descriptive analysis of disease transmission on the data reported from the 1980s and the data collects in 2005-7, using contact event as the unit of analysis, to compare the average number of contacts per case and proportion of household contacts between the two studies. In the 2005-7 study, 86\%** of all reported contacts were classified as household contacts, and the before contact data collection form did ask this question. Therefore, to standardize this variable, we classified household contacts based on their relationship to the case, according to Table 1. We then used the equations provided by Fine et al \cite{Fine1988} to determine the secondary attack rates and crude absolute secondary attack rates during the current and 1980s studies. We used $\chi$\ \textsuperscript{2} tests to determine a significant differences between the characteristics of each study. All analyses were conducted using R.**[cite]

\begin{table} 
\centering
\caption{Catagorization of relationship type into household or non-household contacts.} 
    \begin{tabular}{ l c }
    \toprule
        Relationship & Household Contact \\ 
        \midrule
        Parent & Yes \\ 
        Sibling & Yes \\ 
         &  \\ 
         &  \\ 
         \bottomrule
    \end{tabular} 
\end{table}

To compare the secondary attack rates and effective reproductive numbers ($R_{eff}$) of both surveillance programs, we adapted the equations used by Fine et al \cite{Fine1988}. 
The crude absolute secondary attack rate, $AR_{crude}$, was found by dividing $n_{inf}$, the number of contacts subsequently infected by $n_{total}$, the total number of contacts reported (Equation 1). To determine the $R_{eff}$, we multiplied $AR_{crude}$ by $\bar{n}$, the average number of contacts per case (Equation 2).

\begin{equation}
\label{eqn:equation 1}
AR_{crude}=\frac{n_{inf}}{n_{total}}
\end{equation}

\begin{equation}
\label{eqn:equation 2}
R_{eff}=\sum \left ( \bar{n} \right )AR_{crude}
\end{equation}