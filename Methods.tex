\section{Methods}

From 2005 to 2007, the UCLA-DRC Research Group, in coordination with the DRC’s Ministry of Health (MOH), facilitated an active surveillance program to identify cases of human MPX in the Sankuru district of Kasai Oriental province - the same region as the 1980s WHO program. The active surveillance protocol and results are reported in Rimoin et al \cite{Rimoin2010}. Since 2001, this area of the country has consistently reported the largest incidence of MPX cases and is thus considered an endemic region for MPX \cite{Rimoin2010}. 

Since 2010, MPX has been on the Ministry of Health's list of nationally reportable diseases in the DRC. Thus, all suspected MPX cases in DRC are primarily investigated as mandated by the national program for passive MPX disease surveillance. Consenting cases investigated in the targeted 9 health zones of the Sankuru district who met the  Ministy of Health’s case definition (fever ≥38 ºC, and a vesiculopustular rash), were recruited to participate in this study from 23 November 2005 – 31 November 2007. 

Data from the 1980s WHO active surveillance program were collected from 1981-1986, and were reported in Jezek et al \cite{Jezek1988}.

Ethical approval for this study was obtained from participating institutions (University of California and the Kinshasa School of Public Health), and informed consent was obtained from all participants. All researchers completed required trainings and were approved as key personnel by the institutions. 


