\section{Methods}

From 2005 to 2007, the UCLA-DRC Research Group, in coordination with the DRC’s Ministry of Health (MOH), facilitated an active surveillance program to identify cases of human MPX in the Sankuru district of Kasai Oriental province, in central DRC, the same region as the 1980’s WHO program. The active surveillance protocol and results are reported in Rimoin et al \cite{Rimoin2010}.

Active surveillance took place within 9 health zones of the Sankuru district of Kasai Oriental province in the DRC (Figure 3). The Sankuru has an area of 105,378 km2 and is comprised of dense canopy rainforest, mosaic/gallery forest and savannah \cite{Fuller2011}. The most populous cities, Lodja, Kole and Lomela are connected to supply roads and waterways; however, most villages contain fewer than 1000 people, are difficult to access, and are established in small forest clearings or traditional agricultural land on the forest edge \cite{Rimoin2010}. In 2007, the 9 health zones had an estimated population of 676,839.37 The inhabitants typically rely on subsistence farming and hunting to provide nutrients, with virtually all protein obtained from hunted wildlife, the most common sources being duikers, monkeys, and rodents.2,25,38,39 Since 2001, this area of the country has consistently reported the largest incidence of MPX cases and is thus considered an endemic region for MPX \cite{Rimoin2010}. 

Since 2010, MPX has been on the Ministry of Health's list of nationally reportable diseases in the DRC. Thus, all suspected MPX cases in DRC are primarily investigated as mandated by the national program for passive MPX disease surveillance. Consenting cases investigated in the targeted 9 health zones of the Sankuru district who met the  Ministy of Health’s case definition (fever ≥38 ºC, and a vesiculopustular rash), were recruited to participate in this study from 23 November 2005 – 31 November 2007. 

The MPX surveillance officer in each health zone conducted a clinical examination on all suspected cases, including the collection of crusted scabs and vesicle fluid from active cases, and administered an oral questionnaire in the local dialect, to collect socio-demographic, clinical and exposure data. GPS coordinates were collected for approximately 58% of cases investigated. 

Cases investigated during the study were included in this analysis if their biological sample tested positive for MPX, as determined by PCR assays. 

Ethical approval for this study was obtained from participating institutions (University of California and the Kinshasa School of Public Health), and informed consent was obtained from all participants. All researchers completed required trainings and were approved as key personnel by the institutions. 




