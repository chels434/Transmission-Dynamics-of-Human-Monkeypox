\section{Methods}

From 2005 to 2007, the UCLA-DRC Research Group, in coordination with the DRC’s Ministry of Health (MOH), facilitated an active surveillance program to identify cases of human MPX in the Sankuru district of Kasai Oriental province - the same region as the 1980s WHO program. The active surveillance protocol and results are reported in Rimoin et al \cite{Rimoin2010}. Since 2001, this area of the country has consistently reported the largest incidence of MPX cases and is thus considered an endemic region for MPX \cite{Rimoin2010}. 

Data from the 1980s WHO active surveillance program were collected from 1981-1986, and were reported in Jezek et al \cite{Jezek1988}.

Since 2010, MPX has been on the Ministry of Health's list of nationally reportable diseases in the DRC. Thus, all suspected MPX cases in DRC are primarily investigated as mandated by the national program for passive MPX disease surveillance. Consenting cases investigated in the targeted 9 health zones of the Sankuru district who met the  Ministy of Health’s case definition (fever≥38 ºC, and a vesiculopustular rash), were recruited to participate in this study from 23 November 2005 – 31 November 2007. 

Ethical approval for this study was obtained from participating institutions (University of California and the Kinshasa School of Public Health), and informed consent was obtained from all participants. All researchers completed required trainings and were approved as key personnel by the institutions. 

\subsection{Definitions}
Similar to algorithms described previously,4,9,41 an investigated \textit{source case} was classified as \textit{primary} if it was the first case investigated in an independent outbreak chain, or had no history of contact with a symptomatic human within three weeks before rash onset. Primary cases were assumed to be caused by an animal source. \textit{Onset} is defined as the first day of rash presentation and is considered the first day of infectiousness. Investigated source cases were asked to report all \textit{contacts}, defined as individuals coming into direct contact with a case, in the three weeks after onset. As individuals may have been reported as contacts of multiple cases, each report was considered a \textit{contact event}. \textit{Household contacts} were defined as those residing in the same household as the source case.

A \textit{chain of transmission} includes the primary case, and all subsequent cases stemming from this initial infection. A \textit{cluster} of cases may contain chains arising from multiple primary cases.29 Cases in the same chain or cluster with an onset less than seven days after exposure to the primary case were classified as \textit{co-primary case}s. Any cases with an onset 7-34 days after reported contact with a symptomatic primary case was classified as a \textit{secondary case} of the first generation, and assumed to be caused by a human source. The \textit{generation} of secondary cases was determined by their location within the temporally classified chains of transmission.29

\textit{Secondary attack rates} are defined as the proportion of new cases among contact events, within the known incubation period \cite{Dixon2015}. Attack rates can be compared to the \textit{effective reproductive number} ($R_{eff}$), which is defined as the average number of secondary cases caused by each individual case, with regard to population susceptibility.10,27 This is as opposed to the \textit{basic reproductive number} ($R_{0}$), which is the average number of secondary cases caused by each case in a completely susceptible population.29 