\begin{table} % Add the following just after the closing bracket on this line to specify a position for the table on the page: [h], [t], [b] or [p] - these mean: here, top, bottom and on a separate page, respectively
\centering % Centers the table on the page, comment out to left-justify


\caption{Differences between crude secondary attack rates (ARcrude) and effective reproduction number (Reff) between the active surveillance studies taking place 1981-1986, and 2005-2007. The data from the 1980s is reported as total contact reports, while the more recent study was able to identify individuals who were reported as a contact of multiple cases.}
\begin{tabularx}{400pt}{lXXXXXXXX} % The final bracket specifies the number of columns in the table along with left and right borders which are specified using vertical bars (|); each column can be left, right or center-justified using l, r or c. To specify a precise width, use p{width}, e.g. p{5cm}
\toprule % Top horizontal line
& \multicolumn{4}{c}{1981-1986} & \multicolumn{4}{c}{2005-2007} \\ % Amalgamating several columns into one cell is done using the \multicolumn command as seen on this line
\cmidrule(r){2-5} % Horizontal line spanning less than the full width of the table - you can add (r) or (l) just before the opening curly bracket to shorten the rule on the left or right side
\cmidrule(l){6-9}
 & No. Cases & No. Contacts & AR crude & Reff & No. Cases & No. Contacts & AR crude & Reff\\ % Column names row
\toprule % In-table horizontal line
Total contacts reported & 3686 & 93 & 0.025 & 0.275 & 1861 & 166 & 0.089 & 0.563\\ 
Individuals contacts & - & - & - & - & 1453 & 97 & 0.067 & 0.485\\
\bottomrule % Bottom horizontal line
\end{tabularx}
\label{table1} % A label for referencing this table elsewhere, references are used in text as \ref{label}
\end{table}
