\begin{table} % Add the following just after the closing bracket on this line to specify a position for the table on the page: [h], [t], [b] or [p] - these mean: here, top, bottom and on a separate page, respectively
\centering % Centers the table on the page, comment out to left-justify
\caption{Differences between crude secondary attack rates ($AR_{crude}$) and effective reproduction number ($R_{eff}$) between the active surveillance studies taking place 1981-1986, and 2005-2007. The data from the 1980s is reported as total contact events, while the more recent study was able to identify individuals who were reported as a contact of multiple cases.} % Table caption, can be commented out if no caption is required
\begin{tabular}{p{7cm} p{3cm} p{3cm}} % The final bracket specifies the number of columns in the table along with left and right borders which are specified using vertical bars (|); each column can be left, right or center-justified using l, r or c. To specify a precise width, use p{width}, e.g. p{5cm}
\toprule % Top horizontal line
& \multicolumn{2}{c}{Surveillance Program} \\ % Amalgamating several columns into one cell is done using the \multicolumn command as seen on this line
\cmidrule(l){2-3} % Horizontal line spanning less than the full width of the table - you can add (r) or (l) just before the opening curly bracket to shorten the rule on the left or right side
\textbf{Characteristic} & 1981-1986 & 2005-2007 \\ % Column names row
\midrule % In-table horizontal line
No. contact events & 3686 & 1743 \\ % Content row 1
No. contacts & - & 1349 \\ % Content row 2
No. Source cases & ~ & 282 \\ % Content row 3
Median no. contacts per case (IQR) & 11.8 & 6.2 \\ % Content row 4
$AR_{crude}$ & ~ & ~ \\ % Content row 5
$R_{eff}$ & ~ & ~ \\
\bottomrule % Bottom horizontal line
\end{tabular}
\end{table}