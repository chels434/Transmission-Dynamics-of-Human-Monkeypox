\begin{table} % Add the following just after the closing bracket on this line to specify a position for the table on the page: [h], [t], [b] or [p] - these mean: here, top, bottom and on a separate page, respectively
\centering % Centers the table on the page, comment out to left-justify
\caption{Differences between crude secondary attack rates ($AR_{crude}$) and effective reproduction number ($R_{eff}$) between the active surveillance studies taking place 1981-1986, and 2005-2007.*** The data from the 1980s is reported as total contact reports, while the more recent study was able to identify individuals who were reported as a contact of multiple cases.} % Table caption, can be commented out if no caption is required
\begin{tabular}{lcc} % The final bracket specifies the number of columns in the table along with left and right borders which are specified using vertical bars (|); each column can be left, right or center-justified using l, r or c. To specify a precise width, use p{width}, e.g. p{5cm}
\toprule % Top horizontal line
& \multicolumn{2}{c}{Surveillance Program} \\ % Amalgamating several columns into one cell is done using the \multicolumn command as seen on this line
\cmidrule(l){2-3} % Horizontal line spanning less than the full width of the table - you can add (r) or (l) just before the opening curly bracket to shorten the rule on the left or right side
\textbf{Characteristic} & \textbf{1981-1986} & \textbf{2005-2007} \\ % Column names row
\midrule % In-table horizontal line
No. Contacts & - & 1741 \\ % Content row 1
No. Contact events & 3686 & 1369 \\ % Content row 2
No. Source cases & 338 & 282 \\ % Content row 3
Mean no. contacts per case (range) & 10.2 (1-44) & 6.2 (1-20) \\ % Content row 4
\bottomrule % Bottom horizontal line
\end{tabular}
 % A label for referencing this table elsewhere, references are used in text as \ref{label}
\end{table}


\begin{table} % Add the following just after the closing bracket on this line to specify a position for the table on the page: [h], [t], [b] or [p] - these mean: here, top, bottom and on a separate page, respectively
\centering % Centers the table on the page, comment out to left-justify
\caption{The data from the 1980s is reported as total contact reports, while the more recent study was able to identify individuals who were reported as a contact of multiple cases.} % Table caption, can be commented out if no caption is required
\begin{tabular}{p{3.2cm}p{1.1cm}p{1cm}p{1cm}p{1.1cm}p{1.1cm}p{1cm}p{1cm}p{1.1cm}} % The final bracket specifies the number of columns in the table along with left and right borders which are specified using vertical bars (|); each column can be left, right or center-justified using l, r or c. To specify a precise width, use p{width}, e.g. p{5cm}
\toprule % Top horizontal line
& \multicolumn{4}{c}{\textbf{1981-1986}} & \multicolumn{4}{c}{\textbf{2005-2007}}\\ % Amalgamating several columns into one cell is done using the \multicolumn command as seen on this line
\cmidrule(l){2-5} \cmidrule(l){6-9} % Horizontal line spanning less than the full width of the table - you can add (r) or (l) just before the opening curly bracket to shorten the rule on the left or right side

\textbf{Characteristic} & Total & MPX+ & $AR_{crude}$ & $R_{eff}$ & Total & MPX+ & $AR_{crude}$ & $R_{eff}$ \\ % Column names row
\midrule % In-table horizontal line
No. Source cases & 338 & & & & 689 & 282 & - & - \\ % Content row 3
No. Contacts & - & & & & 1741 & 159 & 0.091 & 0.564 \\ % Content row 1
No. Contact events & 3686 & & & & 1369 & & & \\ [0.1cm]
Mean contacts per case (range) & 10.2 (1-44) & - & - & - & 6.2 (1-20) & - & - & - \\ 
\bottomrule % Bottom horizontal line
\end{tabular}
 % A label for referencing this table elsewhere, references are used in text as \ref{label}
\end{table}

\begin{table} % Add the following just after the closing bracket on this line to specify a position for the table on the page: [h], [t], [b] or [p] - these mean: here, top, bottom and on a separate page, respectively
\footnotesize
\centering % Centers the table on the page, comment out to left-justify
\caption{The data from the 1980s is reported as total contact reports, while the more recent study was able to identify individuals who were reported as a contact of multiple cases.} % Table caption, can be commented out if no caption is required
\begin{tabulary}{1\textwidth}{@{}LCCCCCCCC}
\toprule % Top horizontal line
& \multicolumn{4}{c}{\textbf{1981-1986}} & \multicolumn{4}{c}{\textbf{2005-2007}}\\ % Amalgamating several columns into one cell is done using the \multicolumn command as seen on this line
\cmidrule(l){2-5} \cmidrule(l){6-9} % Horizontal line spanning less than the full width of the table - you can add (r) or (l) just before the opening curly bracket to shorten the rule on the left or right side
\textbf{Characteristic} & Total & MPX+ & $AR_{crude}$ & $R_{eff}$ & Total & MPX+ & $AR_{crude}$ & $R_{eff}$ \\ % Column names row
\midrule % In-table horizontal line
No. Source cases & 338 & & & & 689 & 282 & - & - \\ % Content row 3
No. Contacts & - & & & & 1741 & 159 & 0.091 & 0.564 \\ % Content row 1
No. Contact events & 3686 & & & & 1369 & & & \\ [0.1cm]
Mean contacts per case (range) & 10.2 (1-44) & - & - & - & 6.2 (1-20) & - & - & - \\ 
\bottomrule % Bottom horizontal line
\end{tabulary}
 % A label for referencing this table elsewhere, references are used in text as \ref{label}
\end{table}

