\subsection{3.2	Characteristics of Disease Clusters}

Among the 733 confirmed cases, 435 distinct chains of transmission were identified. Of all chains, 279 (64\%) consisted only of a single primary case with no secondary spread, while 59 (12.8\%) chains contained one or more co-primary cases, but no secondary generations (Table 4). A further 29 (6.7\%) chains were overlapping, meaning they contained one or more co-primary cases as well as at least one generation of secondary cases, making it difficult to distinguish which primary case infected which secondary. There were many chains that may have included additional suspected cases who did not have samples collected because they were either investigated too late, or there was an inadequate supply of materials. For example, in the village of Bolengo in the Kole health zone, 17 cases were investigated over a period of four months. The health worker visited the village three times, 1-1.5 months apart. Due to the long intervals between visits, samples were only able to be collected from 6/17 cases and 11 cases could not be definitively linked to the chain of transmission.  