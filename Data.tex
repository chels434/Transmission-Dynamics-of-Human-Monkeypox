\subsection{Data Collection and Management}
The MPX surveillance officer in each health zone conducted a clinical examination on all suspected cases,
including the collection of crusted scabs and vesicle fluid from active cases and administration of an oral questionnaire in the local dialect, to collect socio-demographic, clinical and exposure data, contact tracing data and GPS coordinates.

At the time of case investigation or during a follow-up visit, patients were interviewed to complete a contact tracing questionnaire reporting contact with symptomatic individuals in the three weeks prior to disease onset  (\textit{before contacts}) and all contacts in the weeks after disease onset (\textit{after contacts}). If a reported contact had previously been investigated as a case, or was investigated at a later date after developing symptoms, the investigating health care worker was asked to record the contact's case ID, and to update this information during follow-up visits to each village. Contact tracing information was recorded on paper and entered into a custom relational database using FileMaker Pro v13 at a later date.

All unique names of both cases and contacts were assigned a new identification code after data entry to protect their identity, and to link the records of identified individuals who had been assigned multiple unique IDs due to typos or data collection errors. Fuzzy matches and similar names were identified using n-gram clustering algorithms provided by the OpenRefine software program. Contacts in each cluster were determined to be the same or different individuals by comparing the associated data collected in each reporting instance. Variables taken into account included the health zone, village, name of the head of household, relationship to the case, and dates of contact. If non-unique names had differing associated data, they were determined to be separate individuals, and were re-assigned new unique IDs. If contacts with unique but similar names were found to have matching associated data, they were re-assigned to the same unique ID **[diagram to show flow?]. The names of all contacts were also compared to the names of all investigated cases, to link any contacts who were also investigated as cases to their case data, regardless of whether their case ID was provided in the contact tracing questionnaire. 

The datasets containing case information and contact tracing information were merged to create a dataset of contact events. Individuals listed as before contacts that had been investigated as cases and all investigated cases reporting after contacts were classified as \textit{sources}, while all individuals listed as \textit{after contacts} and investigated cases reporting \textit{before contacts} were classified as \textit{sinks}.

Within the current study, we used the data on relationship to the case and activities performed with the case to determine whether certain contacts or certain activities had significantly higher risk of becoming infected. 