\section{Results}

\subsection{Characteristics of MPX Cases and Contacts}

* add: GPS coordinates were collected for approximately 58\% of cases investigated. 

During the 2005-2007 active surveillance program in nine health zones of the Sankuru district, 1288 suspected cases were investigated. Samples collected from 1020 cases (79.19\%) underwent laboratory testing, of which 689 cases (67.55\%) were confirmed positive for MPX (MPX+). Of confirmed cases, 282 individuals (40.93\%) reported a total of 1,751 contact events with 1,349 individuals, and an average of 6.21 contacts per source case (range, 1-20). Source cases reported a significantly higher proportion of household contacts, with 85.71\% classified as household compared to 53\% in the 1980s (p-value \textless 0.001).



Overall, 2847 individuals were reported as having close contact with a suspected case, for a total of 3404 contact reports. Of these, 
subsequently, 280 individuals from 397 contact reports were investigated as cases, with 113 individuals and 193 reports confirmed positive for MPX infection. 

\begin{table}
\centering
\begin{tabular}{lccc} 
\toprule
& \textbf{2005-2007} & \textbf{1981-1986} \\
\cmidrule(l){2-3}
\textbf{Type of Contact}  &	\textbf{Total No. Contacts (\%)} &  $AR_{crude}$ &	\textbf{ No. Contacts (\%)} & $AR_{crude}$ \\
\midrule
Household  & (0.84) & ~ & 1420 (62.34\%) & ~\\
Non-Household  & ~ &	(0.16) & 858 (37.66\%) & ~ \\
\midrule
Total &	1573 & ~ & 2278 & 3.03 \\
\bottomrule
\end{tabular}
\caption{Proportion of before and after contacts who were household and non-household contacts}
\label{tab:table1}
\end{table}



The majority of before and after contacts were reported to be siblings (56\% and 51\% respectively), with parents comprising 22\% of after contacts and only 9\% of before
contacts (Figure 4). Individuals external to family were only 4\% of the reported after contacts, however, 13\% of the symptomatic before contacts were external, suggesting instances of household-to-household transmission. Differences in relationship frequencies in before and after contacts may also represent differences in caretaking and susceptibility patterns; for example, no grandparents were indicated as before contacts while 4\% of after contacts were grandparents (Figure 4).

\subsection{Monkeypox Transmission Dynamics}
Of the 293 confirmed MPX cases reporting after contacts, 87 individuals (29.2\%) started a chain of transmission, infecting 96 individual secondary cases, who represented 166 reported contacts due to their contact with multiple cases. Each contact was listed by an average of 1.28 cases (Table 4). The crude absolute secondary attack rate was found to be 0.067 (Table 5).

\subsection{Characteristics of Disease Clusters}
Among the 733 confirmed cases, 435 distinct chains of transmission were identified. Of all chains, 279 (64\%) consisted only of a single primary case with no secondary spread, while 59 (12.8\%) chains contained one or more co-primary cases, but no secondary generations (Table 4). A further 29 (6.7\%) chains were overlapping, meaning they contained one or more co-primary cases as well as at least one generation of secondary cases, making it difficult to distinguish which primary case infected which secondary. There were many chains that may have included additional suspected cases who did not have samples collected because they were either investigated too late, or there was an inadequate supply of materials. For example, in the village of Bolengo in the Kole health zone, 17 cases were investigated over a period of four months. The health worker visited the village three times, 1-1.5 months apart. Due to the long intervals between visits, samples were only able to be collected from 6/17 cases and 11 cases could not be definitively linked to the chain of transmission.

Chains were stratified by whether they contained co-primary cases or not to determine whether there was a significant difference in chain size distribution between chains with co-primaries and chains with just a single primary case. Overall, primary and co-primary cases made up 72.6\% of total cases, and 77.4\% of chains did not spread to secondary generations. Approximately 19.6\% of chains contained one or more co-primary cases, of which 65.9\% did not spread to subsequent generations. This is significantly different than the chains that contained only a single primary case, 80.2\% of which did not spread to secondary generations (Figure 5). Overall, the average chain size was 1.76 cases (range, 1-13) and the longest chain reached six generations. Chains with one primary case contained an average of 1.42 cases (range, 1-13), as opposed to those with co-primaries with 3.14 cases (range, 2-12). 

\subsection{Monkeypox Transmission Dynamics}
Of the 293 confirmed MPX cases reporting after contacts, 87 individuals (29.2\%) started a chain of transmission, infecting 96 individual secondary cases, who represented 166 reported contacts due to their contact with multiple cases. Each contact was listed by an average of 1.28 cases (Table 4). The crude absolute secondary attack rate was found to be 0.067 (Table 5).

Upon stratification into household and non-household contacts, 1224 individuals were listed in 1548 reports of household contacts, and 190 individuals were listed in 269 reports of non-household contacts. We found for both individual contacts and total reported contacts to determine the best method of comparison to the 1980s reports. Individual was 0.070 for household contacts and 0.047 for non-household, while total was 0.091 and 0.089 for household and non-household contacts, respectively (Table 5). Using Equation 2, we found the individual Reff to be 0.419, and the total Reff to be 0.570. These values indicate the average number of secondary cases caused by each case, accounting for differences in attack rates between household and non-household contacts.