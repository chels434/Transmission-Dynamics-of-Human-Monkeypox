\section{Results}

\subsection{Characteristics of MPX Cases and Contacts}

* add: GPS coordinates were collected for approximately 58\% of cases investigated. 

During the 2005-2007 active surveillance program in nine health zones of the Sankuru district, 1288 suspected cases were investigated. Samples collected from 1020 cases (79.19\%) underwent laboratory testing, of which 689 cases (67.55\%) were confirmed positive for MPX (MPX+). Of confirmed cases, 282 individuals (40.93\%) reported a total of 1,751 contact events with 1,349 individuals, and an average of 6.21 contacts per source case (range, 1-20). Source cases reported a significantly higher proportion of household contacts, with 85.71\% classified as household compared to 53\% in the 1980s (p-value < 0.001).



Overall, 2847 individuals were reported as having close contact with a suspected case, for a total of 3404 contact reports. Of these, 
subsequently, 280 individuals from 397 contact reports were investigated as cases, with 113 individuals and 193 reports confirmed positive for MPX infection. 

\begin{table}
\centering
\begin{tabular}{lccc} 
\tabletypesize{\footnotesize}
\toprule
& \textbf{2005-2007} & \textbf{1981-1986} \\
\cmidrule(l){2-3}
\textbf{Type of Contact}  &	\textbf{Total No. Contacts (\%)} &	\textbf{ No. Contacts (\%)} \\
\midrule
Household  & 1549 (0.84) & 834 (0.53) \\
Non-Household  &	295	(0.16) & 739 (0.47) \\
\midrule
Total &	1844 & 1573  \\
\bottomrule
\end{tabular}
\caption{Proportion of before and after contacts who were household and non-household contacts}
\label{tab:table1}
\end{table}



The majority of before and after contacts were reported to be siblings (56\% and 51\% respectively), with parents comprising 22\% of after contacts and only 9\% of before
contacts (Figure 4). Individuals external to family were only 4\% of the reported after contacts, however, 13\% of the symptomatic before contacts were external, suggesting instances of household-to-household transmission. Differences in relationship frequencies in before and after contacts may also represent differences in caretaking and susceptibility patterns; for example, no grandparents were indicated as before contacts while 4\% of after contacts were grandparents (Figure 4).
