\section{Results}

\subsection{Characteristics of MPX Cases and Contacts}

During the 2005-2007 active surveillance program in nine health zones of the Sankuru district, 1407 suspected cases were investigated, of which 733 cases (55.5\%) were confirmed positive for MPX. Of these cases, 293 individuals (39.9\%) reported contact tracing information. with 139 having contact with one or more symptomatic human in the three weeks before illness onset, for a total of 218 before contacts. All 293 individuals reported having close contacts in the three weeks after illness onset, for a total of 1844 after contacts (Table 2). Contacts were partitioned into household or non-household contacts, with household contacts comprising 67\% of before contacts and 84\% of after contacts (Table 2). There were significantly more household after contacts compared to the 1980s, when only 53\% were household (P < 0.001). The average number of before contacts per case was 1.57 (range, 1-6), and the average number of after contacts was 6.27 (range, 1-21) contacts per case. 


\begin{table}[!h]
\resizebox{8cm}{!}
\begin{tabular}{lcccccc} 
\toprule
Type of Contact & Total Before Contacts & Relative Frequency &	Total After Contacts &	Relative Frequency & Total After Contacts (1980s) & Relative Frequency \\
\midrule
Household & 147 & 0.67 & 1549 &	0.84 & 834 & 0.53 \\
Non-Household &	71 & 0.32 &	295	& 0.16 & 739 & 0.47 \\
Total &	218	& &	1844 &	& 1573 & \\
\bottomrule
\end{tabular}
\caption{Proportion of before and after contacts who were household and non-household contacts}
\label{tab:table1}
\end{table}


The majority of before and after contacts were reported to be siblings (56\% and 51\% respectively), with parents comprising 22\% of after contacts and only 9\% of before
contacts (Figure 4). Individuals external to family were only 4\% of the reported after contacts, however, 13\% of the symptomatic before contacts were external, suggesting instances of household-to-household transmission. Differences in relationship frequencies in before and after contacts may also represent differences in caretaking and susceptibility patterns; for example, no grandparents were indicated as before contacts while 4\% of after contacts were grandparents (Figure 4).
