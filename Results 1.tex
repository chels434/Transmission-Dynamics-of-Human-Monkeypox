\section{Results}

\subsection{Characteristics of MPX Cases and Contacts}

* add: GPS coordinates were collected for approximately 58\% of cases investigated. 

During the 2005-2007 active surveillance program in nine health zones of the Sankuru district, 1407 suspected cases were investigated, of which 733 cases (55.5\%) were confirmed positive for MPX. Of these cases, 295 individuals (39.9\%) reported contact tracing information, with 139 case-patients reporting contact events with 218 symptomatic individuals in the three weeks before illness onset. A total of xx (xx\%) before contacts had been investigated as a case and xx (xx\%) were confirmed MPX+. A total of 2543 contact events with 1863 individuals in the three weeks after symptom onset were reported by all 295 case-patients (Table 2). Contacts were partitioned into household or non-household contacts, with household contacts comprising 67\% of before contacts and 84\% of after contacts (Table 2). There were significantly more household after contacts compared to the 1980s, when only 53\% of all contacts were household (P < 0.001). The average number of before contacts per case was 1.57 (range, 1-6), and the average number of after contacts was 6.27 (range, 1-21) contacts per case. 

Overall, 2847 individuals were reported as having close contact with a suspected case, for a total of 3404 contact reports. Of these, 
subsequently, 280 individuals from 397 contact reports were investigated as cases, with 113 individuals and 193 reports confirmed positive for MPX infection. 

\begin{table}[!h]
\begin{tabular}{lp{1.4cm}p{1.4cm}p{1.4cm}p{1.4cm}p{1.4cm}p{1.4cm}} 
\tabletypesize{\footnotesize}
\toprule
Type of Contact & Total Before Contacts (\%)  &	Total After Contacts (\%) &	 Total After Contacts (1980s) (\%) \\
\midrule
Household & 147 (0.67) & 1549 (0.84) & 834 (0.53) \\
Non-Household &	71 (0.32) &	295	(0.16) & 739 (0.47) \\
\midrule
Total &	218	 &	1844 	& 1573  \\
\bottomrule
\end{tabular}
\caption{Proportion of before and after contacts who were household and non-household contacts}
\label{tab:table1}
\end{table}


\begin{deluxetable}{lccccccc}
\centering
\tabletypesize{\footnotesize}
\tablecolumns{8}
\tablewidth{0pt}
 \tablehead{
 \colhead{QSO} \vspace{-0.2cm}& \colhead{$R$ Range} &  & & &  &\colhead{Separation} & \colhead{Result}\\ \vspace{-0.2cm}
  & & \colhead{$\langle f \rangle$} & \colhead{$r_0$} &  \colhead{$\gamma$}& $W$ & & \\
 \colhead{Division} & \colhead{(Mpc/h)} & \colhead{} & \colhead{} & \colhead{}&& \colhead{(\%)} & \colhead{Strength}}
 \startdata
 \vspace{-0.2cm} 1/3 Bright &  & $4.24 \cdot 10^{-4}$ & 6.19 && 96.97 & &\\ \vspace{-0.2cm}
  & [0.3,3] & & & 1.77 & &  96.7 & 1.9$\sigma$ \\
   2/3 Dim & & $4.26 \cdot 10^{-4}$ & 4.48 & & 52.77
 \enddata
 \vspace{-0.8cm}

\end{deluxetable}



The majority of before and after contacts were reported to be siblings (56\% and 51\% respectively), with parents comprising 22\% of after contacts and only 9\% of before
contacts (Figure 4). Individuals external to family were only 4\% of the reported after contacts, however, 13\% of the symptomatic before contacts were external, suggesting instances of household-to-household transmission. Differences in relationship frequencies in before and after contacts may also represent differences in caretaking and susceptibility patterns; for example, no grandparents were indicated as before contacts while 4\% of after contacts were grandparents (Figure 4).
