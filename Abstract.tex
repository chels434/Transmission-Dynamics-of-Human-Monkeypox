\section{Abstract}

Reports from the first monkeypox (MPX) active surveillance program in the Democratic Republic of Congo (DRC) in the 1980s determined that the disease was not of epidemic potential, with R\textsubscript{0}\textless1. However, during an active surveillance program from 2005 to 2007, researchers found a 20-fold increase in prevalence over the past 30 years. The purpose of this study was to analyze the contact tracing data from 2005-07 and compare characteristics to those of the 1980s, to assess the current R\textsubscript{0} of MPX, and to determine whether there has been a significant change since the 1980s. Contact tracing information and samples of lesions from active cases were collected. Samples were screened by PCR and positive cases were classified by generation and grouped into chains of transmission according to date of rash onset, contact history, and village. R\textsubscript{0} was determined using calculations provided in the 1980s study and chain size distribution was compared. Of 1407 suspected cases of MPX investigated in 2005-07, 733 were confirmed MPX positive, and 293 provided contact tracing information with an average of 6.28 (range, 1-21) contacts each. Among confirmed cases, 435 distinct chains of transmission were identified with an average chain size of 5.3 cases (range, 1-13), and the longest reaching six generations. The crude absolute secondary attack rate (AR\textsubscript{abs}) was 0.067, with a crude secondary attack rate (AR\textsubscript{crude} of 0.070 and 0.49 among household and non-household contacts, respectively. R\textsubscript{0} was found to be between 0.425-0.576 as opposed to 0.815. Contact characteristics and types of contacts differed from those of the 1980s program. This analysis found a higher crude secondary attack rate, but a lower number of contacts on average, as well as a smaller difference in the attack rates within household and non-household members. This could be the result of a higher proportion of unvaccinated contacts, or that the virus is better able to transmit between humans with a more limited amount of contact. Further analysis of R\textsubscript{0} should be continued in order to evaluate the many variables potentially involved in the transmission differences. 
